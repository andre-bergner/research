\documentclass[journal]{IEEEtran}
\usepackage[dvips]{graphicx}

% Some very useful LaTeX packages include:
% (uncomment the ones you want to load)




% *** GRAPHICS RELATED PACKAGES ***
%
%\ifCLASSINFOpdf
  % \usepackage[pdftex]{graphicx}
  % declare the path(s) where your graphic files are
  % \graphicspath{{../pdf/}{../jpeg/}}
  % and their extensions so you won't have to specify these with
  % every instance of \includegraphics
  % \DeclareGraphicsExtensions{.pdf,.jpeg,.png}
%\else
  % or other class option (dvipsone, dvipdf, if not using dvips). graphicx
  % will default to the driver specified in the system graphics.cfg if no
  % driver is specified.
  % \usepackage[dvips]{graphicx}
  % declare the path(s) where your graphic files are
  % \graphicspath{{../eps/}}
  % and their extensions so you won't have to specify these with
  % every instance of \includegraphics
  % \DeclareGraphicsExtensions{.eps}
%\fi
% graphicx was written by David Carlisle and Sebastian Rahtz. It is
% required if you want graphics, photos, etc. graphicx.sty is already
% installed on most LaTeX systems. The latest version and documentation can
% be obtained at: 
% http://www.ctan.org/tex-archive/macros/latex/required/graphics/
% Another good source of documentation is "Using Imported Graphics in
% LaTeX2e" by Keith Reckdahl which can be found as epslatex.ps or
% epslatex.pdf at: http://www.ctan.org/tex-archive/info/
%
% latex, and pdflatex in dvi mode, support graphics in encapsulated
% postscript (.eps) format. pdflatex in pdf mode supports graphics
% in .pdf, .jpeg, .png and .mps (metapost) formats. Users should ensure
% that all non-photo figures use a vector format (.eps, .pdf, .mps) and
% not a bitmapped formats (.jpeg, .png). IEEE frowns on bitmapped formats
% which can result in "jaggedy"/blurry rendering of lines and letters as
% well as large increases in file sizes.
%
% You can find documentation about the pdfTeX application at:
% http://www.tug.org/applications/pdftex



% *** MATH PACKAGES ***
%
\usepackage[cmex10]{amsmath}
\usepackage{amssymb}
\usepackage{amsfonts}
% A popular package from the American Mathematical Society that provides
% many useful and powerful commands for dealing with mathematics. If using
% it, be sure to load this package with the cmex10 option to ensure that
% only type 1 fonts will utilized at all point sizes. Without this option,
% it is possible that some math symbols, particularly those within
% footnotes, will be rendered in bitmap form which will result in a
% document that can not be IEEE Xplore compliant!
%
% Also, note that the amsmath package sets \interdisplaylinepenalty to 10000
% thus preventing page breaks from occurring within multiline equations. Use:
%\interdisplaylinepenalty=2500
% after loading amsmath to restore such page breaks as IEEEtran.cls normally
% does. amsmath.sty is already installed on most LaTeX systems. The latest
% version and documentation can be obtained at:
% http://www.ctan.org/tex-archive/macros/latex/required/amslatex/math/



% *** ALIGNMENT PACKAGES ***
%
%\usepackage{array}
% Frank Mittelbach's and David Carlisle's array.sty patches and improves
% the standard LaTeX2e array and tabular environments to provide better
% appearance and additional user controls. As the default LaTeX2e table
% generation code is lacking to the point of almost being broken with
% respect to the quality of the end results, all users are strongly
% advised to use an enhanced (at the very least that provided by array.sty)
% set of table tools. array.sty is already installed on most systems. The
% latest version and documentation can be obtained at:
% http://www.ctan.org/tex-archive/macros/latex/required/tools/




% IEEEtran contains the IEEEeqnarray family of commands that can be used to
% generate multiline equations as well as matrices, tables, etc., of high
% quality.


%\usepackage{eqparbox}
% Also of notable interest is Scott Pakin's eqparbox package for creating
% (automatically sized) equal width boxes - aka "natural width parboxes".
% Available at:
% http://www.ctan.org/tex-archive/macros/latex/contrib/eqparbox/





% *** SUBFIGURE PACKAGES ***
%\usepackage[tight,footnotesize]{subfigure}
% subfigure.sty was written by Steven Douglas Cochran. This package makes it
% easy to put subfigures in your figures. e.g., "Figure 1a and 1b". For IEEE
% work, it is a good idea to load it with the tight package option to reduce
% the amount of white space around the subfigures. subfigure.sty is already
% installed on most LaTeX systems. The latest version and documentation can
% be obtained at:
% http://www.ctan.org/tex-archive/obsolete/macros/latex/contrib/subfigure/
% subfigure.sty has been superceeded by subfig.sty.



%\usepackage[caption=false]{caption}
%\usepackage[font=footnotesize]{subfig}
% subfig.sty, also written by Steven Douglas Cochran, is the modern
% replacement for subfigure.sty. However, subfig.sty requires and
% automatically loads Axel Sommerfeldt's caption.sty which will override
% IEEEtran.cls handling of captions and this will result in nonIEEE style
% figure/table captions. To prevent this problem, be sure and preload
% caption.sty with its "caption=false" package option. This is will preserve
% IEEEtran.cls handing of captions. Version 1.3 (2005/06/28) and later 
% (recommended due to many improvements over 1.2) of subfig.sty supports
% the caption=false option directly:
%\usepackage[caption=false,font=footnotesize]{subfig}
%
% The latest version and documentation can be obtained at:
% http://www.ctan.org/tex-archive/macros/latex/contrib/subfig/
% The latest version and documentation of caption.sty can be obtained at:
% http://www.ctan.org/tex-archive/macros/latex/contrib/caption/




% *** FLOAT PACKAGES ***
%
%\usepackage{fixltx2e}
% fixltx2e, the successor to the earlier fix2col.sty, was written by
% Frank Mittelbach and David Carlisle. This package corrects a few problems
% in the LaTeX2e kernel, the most notable of which is that in current
% LaTeX2e releases, the ordering of single and double column floats is not
% guaranteed to be preserved. Thus, an unpatched LaTeX2e can allow a
% single column figure to be placed prior to an earlier double column
% figure. The latest version and documentation can be found at:
% http://www.ctan.org/tex-archive/macros/latex/base/



%\usepackage{stfloats}
% stfloats.sty was written by Sigitas Tolusis. This package gives LaTeX2e
% the ability to do double column floats at the bottom of the page as well
% as the top. (e.g., "\begin{figure*}[!b]" is not normally possible in
% LaTeX2e). It also provides a command:
%\fnbelowfloat
% to enable the placement of footnotes below bottom floats (the standard
% LaTeX2e kernel puts them above bottom floats). This is an invasive package
% which rewrites many portions of the LaTeX2e float routines. It may not work
% with other packages that modify the LaTeX2e float routines. The latest
% version and documentation can be obtained at:
% http://www.ctan.org/tex-archive/macros/latex/contrib/sttools/
% Documentation is contained in the stfloats.sty comments as well as in the
% presfull.pdf file. Do not use the stfloats baselinefloat ability as IEEE
% does not allow \baselineskip to stretch. Authors submitting work to the
% IEEE should note that IEEE rarely uses double column equations and
% that authors should try to avoid such use. Do not be tempted to use the
% cuted.sty or midfloat.sty packages (also by Sigitas Tolusis) as IEEE does
% not format its papers in such ways.


%\ifCLASSOPTIONcaptionsoff
%  \usepackage[nomarkers]{endfloat}
% \let\MYoriglatexcaption\caption
% \renewcommand{\caption}[2][\relax]{\MYoriglatexcaption[#2]{#2}}
%\fi
% endfloat.sty was written by James Darrell McCauley and Jeff Goldberg.
% This package may be useful when used in conjunction with IEEEtran.cls'
% captionsoff option. Some IEEE journals/societies require that submissions
% have lists of figures/tables at the end of the paper and that
% figures/tables without any captions are placed on a page by themselves at
% the end of the document. If needed, the draftcls IEEEtran class option or
% \CLASSINPUTbaselinestretch interface can be used to increase the line
% spacing as well. Be sure and use the nomarkers option of endfloat to
% prevent endfloat from "marking" where the figures would have been placed
% in the text. The two hack lines of code above are a slight modification of
% that suggested by in the endfloat docs (section 8.3.1) to ensure that
% the full captions always appear in the list of figures/tables - even if
% the user used the short optional argument of \caption[]{}.
% IEEE papers do not typically make use of \caption[]'s optional argument,
% so this should not be an issue. A similar trick can be used to disable
% captions of packages such as subfig.sty that lack options to turn off
% the subcaptions:
% For subfig.sty:
% \let\MYorigsubfloat\subfloat
% \renewcommand{\subfloat}[2][\relax]{\MYorigsubfloat[]{#2}}
% For subfigure.sty:
% \let\MYorigsubfigure\subfigure
% \renewcommand{\subfigure}[2][\relax]{\MYorigsubfigure[]{#2}}
% However, the above trick will not work if both optional arguments of
% the \subfloat/subfig command are used. Furthermore, there needs to be a
% description of each subfigure *somewhere* and endfloat does not add
% subfigure captions to its list of figures. Thus, the best approach is to
% avoid the use of subfigure captions (many IEEE journals avoid them anyway)
% and instead reference/explain all the subfigures within the main caption.
% The latest version of endfloat.sty and its documentation can obtained at:
% http://www.ctan.org/tex-archive/macros/latex/contrib/endfloat/
%
% The IEEEtran \ifCLASSOPTIONcaptionsoff conditional can also be used
% later in the document, say, to conditionally put the References on a 
% page by themselves.





% *** PDF, URL AND HYPERLINK PACKAGES ***
%
%\usepackage{url}
% url.sty was written by Donald Arseneau. It provides better support for
% handling and breaking URLs. url.sty is already installed on most LaTeX
% systems. The latest version can be obtained at:
% http://www.ctan.org/tex-archive/macros/latex/contrib/misc/
% Read the url.sty source comments for usage information. Basically,
% \url{my_url_here}.





% *** Do not adjust lengths that control margins, column widths, etc. ***
% *** Do not use packages that alter fonts (such as pslatex).         ***
% There should be no need to do such things with IEEEtran.cls V1.6 and later.
% (Unless specifically asked to do so by the journal or conference you plan
% to submit to, of course. )




% correct bad hyphenation here
\hyphenation{wave-let}


\newcommand	{\WT}	{$\mathcal{W}_{\!\psi}x$ }
\newcommand	{\z}		{z^{-1}}

\newcommand	{\be}	{\begin{equation}}
\newcommand	{\ee}	{\end{equation}}





\begin{document}
\title {The fastest possible realtime Gabor transform}
\author{A. Bergner \thanks{Berlin, Germany} }
\date  {\today}
\maketitle




\begin{abstract}

The Abstract...

\end{abstract}

\begin{IEEEkeywords}
  gabor transform, realtime, fast implementation
\end{IEEEkeywords}



%================================================================================================
%================================================================================================


  \section{Inroduction}
  \label{sec:intro}


\IEEEPARstart{T}{he} Gabor transform (GT) is a standard tool in many signal
analysis and image processing applications, in particular when dealing with
instationary signals or signals with inherent multiple timescales (e.g. EEG
analysis, music/audio, geophysics, etc.).

Interesting property (difficult to impossible with STFT):  high precession phase retrieval and instantaneous frequency



The GT $\mathcal{G}x$ of a signal $x(t)$ is given by the inner product
\begin{equation} \label{def:GT}
  \mathcal{G}x\,(\omega,\tau) \ := \ \langle \epsilon_{\omega\tau} , x \rangle
    \ = \ \int_{\mathbb R}\!\! dt \; \epsilon_{\omega\tau} (t)\, x(t)
\end{equation}
The Gabor atoms $\epsilon_{\omega t}$ form a two-parametric family
\begin{equation}
  \mathcal{E} \ = \ \left\{ \epsilon_{\omega\tau}(t) =  e^{i\omega (t-\tau)} e^{-(t-\tau)^2}
                      \ \middle| \   \omega,\tau \in \mathbb{R} \right\}
\end{equation}
of a by $\tau$ translated gaussian windowed complex oscillations.

The GT has the disadvantage that the inner product (\ref{def:GT}) has to be computed for every
point $(\omega,\tau)$ in the time-frequency plane.
If (\ref{def:GT}) is computed in this direct fashion the computational amount is enormous
if a high resolution approximation of the GT is needed.
Fortunately (\ref{def:GT}) can be rewritten as a convolution operation which in turn can be
computed utilizing the fast convolution algorithm, which at least reduces the computational
complexity from ${\mathcal O}(N^2)$ to ${\mathcal O}(N log N )$

Based on the idea of convolution is: Filterbanks (IIR as convolution) \cite{Unser1994}

Other methods: interpolating a STFT, sliding FFT

....

The method presented here is inspired by the functionality of the cochlea inside the inner ear.
From that a coupled filter bank is derived which outperforms any known method for computing
an discrete approximation of the continuous GT. My method has ${\mathcal O}(N)$ complexity
and needs 3 additions and 3 multiplications per output sample for real and imaginary part respectively,
which is even less then the number of operation needed for just linear interpolating a STFT.
Further no pre- or post-processing is needed with this method.




% - - - - - - - - - - - - - - - - - - - - - - - - - - - - - - - - - - - - - - - - - - - - - - - - - -


  \section{Derivation of the coupled filter bank}
  \label{sec:construction}

The construction of the filter bank was motivated by the functionality of the cochlea inside the inner ear.
I will not go into all the details of the inner ear and the functionality of the cochlea in particular (see
refs for more details). However, given that the cochlea is a membrane with a heterogenous frequency response
along its axis (tonotopic map) the linear key properties can be modeled by a damped inhomogeneous complex diffusion equation (similar to a Schr\"odinger equation), namely
\be \label{pde}
	\partial_t u = (\gamma + i x + \varkappa \partial_x^2 ) u  +  f(t),
\ee
where $u(x,t)$ is the deviation of the membrane from it point of rest, $\gamma$ is the damping coefficient,
$\omega x$ is the inhomogeneous frequency term and $\varkappa$ the membrane's coupling strength (TODO correct term).

The solution of the unforced system ($f \equiv 0$) can be found analytically. We use the ansatz
\be \label{ansatz1}
	u(x,t) = e^{at + bt^2 + ct^3} e^{ixt}
\ee
Inserting (\ref{ansatz1}) into (\ref{pde}) one obtains
$$
	a  +  2 b t  +  3 c t^2  +  ix \; = \; \gamma + i x - \varkappa t^2,
$$
and after collecting all terms the coefficients can be determined to be $a=\gamma$, $b=0$, and $\varkappa = -3c$
which leaves us with the final solution
\be \label{pde-solution}
	u(x,t) = e^{\gamma t - \varkappa t^3/3} e^{ixt}
\ee
(TODO describe solution: initial condition, cut-off gaussian-like impulse window response)



Partially discretizing Eq. (\ref{pde}) in its spatial component by simply replacing the Laplacian by its finite
difference version we obtain the difference-differential equation
\be \label{dde}
	\dot{u}_n = (\gamma + i n) u_n  +  \varkappa ( u_{n-1} - 2 u_n + u_{n+1} )  +  f(t),
\ee
This one can be solved with the ansatz
\be \label{ansatz2}
	u_n(t) = e^{a t  +  b \sin t} e^{int}.
\ee
Inserting (\ref{ansatz2}) in (\ref{dde}) we get
$$
     a + b \cos t + i n \; = \; \gamma + i n + 2\varkappa\; ( \cos t - 1 ).
$$
By comparing all coefficients one obtains $a = \gamma - 2 \varkappa$ and $b = 2\varkappa$ and together with
(\ref{ansatz2}) the full solution reads
\be
	u_n(t) = e^{(\gamma - 2 \varkappa) t  +  2\varkappa \sin t} e^{int}.
\ee
(TODO explain different to prev. solution)




Simple linear models of cochlea in the inner ear are in essence a filterbank 

\be \label{eq:discrete_laplacian}
\begin{pmatrix}
      1  & -1 & & & \multicolumn{2}{c}{\text{\kern0.5em\smash{\raisebox{-2ex}{\Huge 0}}}} \\
      -1 &  2 & -1 \\
         & & \dots & & \\
	     & & -1 &  2 & -1 \\
      \multicolumn{2}{c}{\text{\kern-1.5em\smash{\raisebox{0.25ex}{\Huge 0}}}} & & -1 & 1
    \end{pmatrix}
\ee


The power of the method is .. to the fact that a high-dimensional system copmutes all channels
simultanously due to high symmwetry



%================================================================================================


  \section{Reconstruciot}
  \label{sec:reconst}
 
- almost perfect reconstruction --> konverges fast
- plot relative reconst error ( dev - mean )/ mean  oder so


%================================================================================================


  \section{Propperties}
  \label{sec:propperties}
 
- Gabor Projector
- reproducing kernel
- theoretically --> discrete version reliased here


%================================================================================================


  \section{Approximation Error}
  \label{sec:error}

There is an increasing dilation error with increasing center frequency. However, this dilation error
can be controlled. First we need to quatify the error. There are different
quantities in which one might be interested. The most important quantity is the Q-factor, the ratio
between center frequency and band-width of the wavelet, which should remain constant under rescaling.




%================================================================================================


  \section{Speed Comparison}
  \label{sec:speed}

%  - compare scale-wise
%  - compare for different freq-localisations
%  - create wavelet from filter impulse-response
%    -> direct computation via convolution
%    -> computation via FFTW
%  - comparison with  à trous, IEEE-versions, ... ? Perhaps based on arithmetic complexity
%    here main point: localisation (doesn't matter with my wavelet)

%================================================================================================


  \section{Wavelet Properties}
  \label{sec:properties}


%================================================================================================


\section{Conclusions}
\label{sec:conclusions}


%================================================================================================

\appendices
\section{Python Code}
I present a small a python script that computes a Gabor approximation


%================================================================================================

\begin{thebibliography}{99}

\bibitem{Gabor1946}
  \textit{Theory of Communication},
  J. IEEE, Vol. 93, No. 26. (November 1946), pp. 429-457

\bibitem{Unser1994}
  \textit{Fast Gabor-Like Windowed Fourier and Continuous Wavelet Transforms},
  IEEE Signal Processing Lett., Vol. 1, No. 5. (May 1946), pp. 76-79

\end{thebibliography}

%================================================================================================
%================================================================================================
%================================================================================================




% An example of a floating figure using the graphicx package.
% Note that \label must occur AFTER (or within) \caption.
% For figures, \caption should occur after the \includegraphics.
% Note that IEEEtran v1.7 and later has special internal code that
% is designed to preserve the operation of \label within \caption
% even when the captionsoff option is in effect. However, because
% of issues like this, it may be the safest practice to put all your
% \label just after \caption rather than within \caption{}.
%
% Reminder: the "draftcls" or "draftclsnofoot", not "draft", class
% option should be used if it is desired that the figures are to be
% displayed while in draft mode.
%
%\begin{figure}[!t]
%\centering
%\includegraphics[width=2.5in]{myfigure}
% where an .eps filename suffix will be assumed under latex, 
% and a .pdf suffix will be assumed for pdflatex; or what has been declared
% via \DeclareGraphicsExtensions.
%\caption{Simulation Results}
%\label{fig_sim}
%\end{figure}

% Note that IEEE typically puts floats only at the top, even when this
% results in a large percentage of a column being occupied by floats.


% An example of a double column floating figure using two subfigures.
% (The subfig.sty package must be loaded for this to work.)
% The subfigure \label commands are set within each subfloat command, the
% \label for the overall figure must come after \caption.
% \hfil must be used as a separator to get equal spacing.
% The subfigure.sty package works much the same way, except \subfigure is
% used instead of \subfloat.
%
%\begin{figure*}[!t]
%\centerline{\subfloat[Case I]\includegraphics[width=2.5in]{subfigcase1}%
%\label{fig_first_case}}
%\hfil
%\subfloat[Case II]{\includegraphics[width=2.5in]{subfigcase2}%
%\label{fig_second_case}}}
%\caption{Simulation results}
%\label{fig_sim}
%\end{figure*}
%
% Note that often IEEE papers with subfigures do not employ subfigure
% captions (using the optional argument to \subfloat), but instead will
% reference/describe all of them (a), (b), etc., within the main caption.


% An example of a floating table. Note that, for IEEE style tables, the 
% \caption command should come BEFORE the table. Table text will default to
% \footnotesize as IEEE normally uses this smaller font for tables.
% The \label must come after \caption as always.
%
%\begin{table}[!t]
%% increase table row spacing, adjust to taste
%\renewcommand{\arraystretch}{1.3}
% if using array.sty, it might be a good idea to tweak the value of
% \extrarowheight as needed to properly center the text within the cells
%\caption{An Example of a Table}
%\label{table_example}
%\centering
%% Some packages, such as MDW tools, offer better commands for making tables
%% than the plain LaTeX2e tabular which is used here.
%\begin{tabular}{|c||c|}
%\hline
%One & Two\\
%\hline
%Three & Four\\
%\hline
%\end{tabular}
%\end{table}


% Note that IEEE does not put floats in the very first column - or typically
% anywhere on the first page for that matter. Also, in-text middle ("here")
% positioning is not used. Most IEEE journals use top floats exclusively.
% Note that, LaTeX2e, unlike IEEE journals, places footnotes above bottom
% floats. This can be corrected via the \fnbelowfloat command of the
% stfloats package.





% if have a single appendix:
%\appendix[Proof of the Zonklar Equations]
% or
%\appendix  % for no appendix heading
% do not use \section anymore after \appendix, only \section*
% is possibly needed

% use appendices with more than one appendix
% then use \section to start each appendix
% you must declare a \section before using any
% \subsection or using \label (\appendices by itself
% starts a section numbered zero.)
%


% trigger a \newpage just before the given reference
% number - used to balance the columns on the last page
% adjust value as needed - may need to be readjusted if
% the document is modified later
%\IEEEtriggeratref{8}
% The "triggered" command can be changed if desired:
%\IEEEtriggercmd{\enlargethispage{-5in}}

% references section

% can use a bibliography generated by BibTeX as a .bbl file
% BibTeX documentation can be easily obtained at:
% http://www.ctan.org/tex-archive/biblio/bibtex/contrib/doc/
% The IEEEtran BibTeX style support page is at:
% http://www.michaelshell.org/tex/ieeetran/bibtex/
%\bibliographystyle{IEEEtran}
% argument is your BibTeX string definitions and bibliography database(s)
%\bibliography{IEEEabrv,../bib/paper}
%
% <OR> manually copy in the resultant .bbl file
% set second argument of \begin to the number of references
% (used to reserve space for the reference number labels box)



\end{document}


